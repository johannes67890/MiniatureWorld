% !TeX root = diary.tex


\documentclass[11pt]{article}
\usepackage{lingmacros}
\usepackage{tree-dvips}
\usepackage[utf8]{inputenc}
\usepackage{fancyhdr}
\usepackage{listings}
\usepackage{subfig}
\usepackage{xcolor}
\usepackage{geometry}
\geometry{
a4paper,
total={170mm,257mm},
left=15mm,
top=20mm,
}
\usepackage{graphicx}
\usepackage{wrapfig}
\usepackage{amsmath}
\usepackage{hyperref}
\definecolor{link}{rgb}{0,0,215}
\hypersetup{
    colorlinks=true,
    linkcolor=link,
    filecolor=link,      
    urlcolor=link,
    pdfpagemode=FullScreen,
}

\graphicspath{ {./img} }

\definecolor{comment}{rgb}{0,0.45,0}
\definecolor{codegray}{rgb}{0.5,0.5,0.5}
\definecolor{codepurple}{rgb}{0.58,0,0.82}
\definecolor{backcolour}{rgb}{0.95,0.95,0.92}
\lstdefinestyle{CodeStyle}{
    backgroundcolor=\color{backcolour},   
    commentstyle=\color{comment},
    keywordstyle=\color{magenta},
    numberstyle=\tiny\color{codegray},
    stringstyle=\color{codepurple},
    basicstyle=\ttfamily\footnotesize,
    breakatwhitespace=false,         
    breaklines=true,                 
    captionpos=b,                    
    keepspaces=true,                 
    numbers=left,                    
    numbersep=5pt,                  
    showspaces=false,                
    showstringspaces=false,
    showtabs=false,                  
    tabsize=2
}
\lstset{style=CodeStyle}
\lstdefinelanguage{JavaScript}{
  keywords={typeof, new, true, false, catch, function, return, null, catch, switch, var, if, in, while, do, else, case, break},
  keywordstyle=\color{purple}\bfseries,
  ndkeywords={class, export, const, var, let, boolean, throw, implements, import, this, !!, !=, ===, ;},
  ndkeywordstyle=\color{blue}\bfseries,
  identifierstyle=\color{black},
  sensitive=false,
  comment=[l]{//},
  morecomment=[s]{/*}{*/},
  commentstyle=\color{comment}\ttfamily,
  stringstyle=\color{orange}\ttfamily,
  morestring=[b]',
  morestring=[b]"
}
\usepackage[]{mdframed}

%%%%%%% Document begin %%%%%%%%%%%
\begin{document}
    \title{Peergrade assignment 1}
    \author{Author anonymous} % Do not include your name, as Peergrading is anonymous 
    \date{\today}

    \section*{Week 1. }
    This week we have implemented the following features:
    \begin{itemize}
        \item \textbf{TestReader} - reads test files for information
        \item \textbf{Distributer} - distributes testfile paths
        \item \textbf{Borrow} - Rabbit holes
        \item \textbf{Grass} - Sweat green grass for the simulation
        \item \textbf{Rabbit} - A cute rabbit with various functions
        \item \textbf{BabyRabbit} - A cute baby rabbit (inherets from Rabbit)
    \end{itemize}

    \subsection*{TestReader}
    The TestReader reads the test files for information out the types and amount of entities in the simulation. The TestReader is being used along side the \textbf{Distributer}, to distribute the test files to the simulation.
    There are numores functions in the TestReader, so it is easy to get the information out of the test files. 
    The TestReaders main function is \textbf{getMap} function, that returns a map of the test file with the corrusponting entity type and thier amount. 
    TestReader also has a corrusponting test file, that is used to test the TestReader.
    \\
    \textbf{TestReader} has the following functions:
    \begin{mdframed}
        getFilePath(); getFileContent(); getFileContentString(); getMap(); getRandomIntervalNumber(); getWorldSize(); isNumeric();
    \end{mdframed}
    \subsection*{Distributer}
    The Distributer distributes the test files path for easy use in the simulation. The Distributer is a simple enum that stores all the test file paths and has some functions to get the test file paths.
    \\
    \textbf{Distributer} has the following functions:
    \begin{mdframed}
        getUrl();
    \end{mdframed}
    \subsection*{Borrow}
    The Borrow is a simple class that is used to create rabbit holes in the simulation. The does don't have a constructor and is only used to create and dispay rabit holes. This could be interface 
    \\
    \textbf{Borrow} has the following functions:
    \begin{mdframed}
        \textit{none}
    \end{mdframed}
    \subsection*{Grass}
    Grass is a class with diffrent functions that is used for the behavior of the grass object. One of the functions is the \textbf{spread} function. The function is used to spread the grass object to the surrounding tiles with a 10$\%$ chance of the grass spreading. 
    \\
    \textbf{Grass} has the following functions:
    \begin{mdframed}
        act($World$ world); spread($World$ world)
    \end{mdframed}
    \newpage
    \subsection*{Rabbit}
    The Rabbit class constructs cute rabbits to the simulation with normal rabbit behaviors. Rabbits can move and eat grass, but unfortunately die of starvation as well. 
    Most of the development time was spend on the Rabbit class. all requirements has not been fully developed this week, but the Rabbit class is somewhat functional. 
    There has also been made a baby rabbit class, that is used to create baby rabbits for the reproduction function.
    \\
    \textbf{Rabbit} has the following functions:
    \begin{mdframed}
        act($World$ world); digBorrow($World$ world); move($World$ world); die($World$ world); eat($World$ world); reproduce($World$ world); getRandomSurroundingTile($World$ world); 
    \end{mdframed}
    \subsection*{Baby Rabbit}
    The Baby Rabbit class is a class that is used to create baby rabbits for the reproduction function. The baby rabbit class is a simple class that is used to create baby rabbits. This class is not fully functional after this week, but will be worked on.
    \\
    \textbf{Baby Rabbit} has the following functions:
    \begin{mdframed}
        @override act($World$ world); 
    \end{mdframed}

\end{document}
